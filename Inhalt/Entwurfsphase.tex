% !TEX root = ../Projektdokumentation.tex
\section{Entwurfsphase}
\label{sec:Entwurfsphase}

\subsection{Zielplattform}
\label{sec:Zielplattform}
Für die 64 Bit Version von Windows 10 wird ein Prozessor mit mindestens 1 GHz Arbeitsleistung oder ein {\acs{SoC}} benötigt. Die Mindestkapazität des RAM liegt bei 2 GB, der Festplattenspeicher muss mindestens 32 GB groß sein. Die Grafikkarte muss über DirectX 9 oder höher mit einem {\acs{WDDM}} 1.0 Treiber verfügen.

\subsection{Qualitätssicherung}
\label{sec:Qualitätssicherung}
Zur Sicherung der Qualität wird die Zuverlässigkeit der Software getestet. Beide Seiten der Schnittstelle müssen zu jeder Zeit erreichbar sein und auf Anfragen reagieren bzw. Anfragen senden können. Gleichzeitig muss in Fehlerfällen reagiert werden und eine weitere Bereitstellung der Dienste gesichert sein. Zur Gewährleistung dessen wird die Software im fertigen Zustand Stresstests mit großen Datenmengen und absichtlich verursachten Fehlerfällen ausgesetzt. 

Zur Gewährleistung der Einsatzbereitschaft ist es erforderlich, die einzubindenden Bibliotheken eindeutig zu dokumentieren um die Menge der möglichen Implementierungsfehler so klein wie möglich zu halten.