% !TEX root = ../Projektdokumentation.tex
\section{Einführungsphase}
\label{sec:Einfuehrungsphase}

\subsection{Einrichtungsvorraussetzungen}
\label{sec:Einrichtungsvorraussetzungen}

\subsubsection{Hardware}
\label{sec:HardwareEinrichtung}
Zum Betrieb der Software ist folgende Hardware erforderlich:

\begin{itemize}
	\item 1 Arbeitsplatz-PC (inkl. Bildschirme etc.)
	\item Steuereinheit am Diagnostikgerät
	\item IP-Netzwerk in dem beide Geräte enthalten sind
\end{itemize}

\subsubsection{Software}
\label{sec:SoftwareEinrichtung}
Es wird zum Betrieb der Software eine 64 Bit Version von Windows 7/8/10/11 gefordert, auf der das .NET Framework, mindestens Version 4.8\footnote{.NET Framework: \url{https://dotnet.microsoft.com/download/dotnet-framework/net48}}, installiert ist. Des weiteren muss auf der Serverseite der Port zur Abfrage im Netzwerk frei gegeben werden, sodass die Clientseite die Verbrauchsdaten abfragen kann.

\subsection{Installation}
\label{sec:Installation}
Zur Verwendung der Software auf einem Gerät muss vorher das .NET Framework in der Version 4.8 installiert werden, welches von der offiziellen Microsoft Seite heruntergeladen werden kann. Die Anwendung wurde inklusive aller benötigten Dateien auf einen lokalen {\acs{GitLab}}-Server hochgeladen. Zur Nutzung müssen somit nur die Daten heruntergeladen werden und die Software kann genutzt werden.

Ein eigenständiges Installationsprogramm existiert für diese Anwendung nicht, es reicht der reine Kopiervorgang aus.