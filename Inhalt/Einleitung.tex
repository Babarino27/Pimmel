% !TEX root = ../Projektdokumentation.tex
\section{Einleitung}
\label{sec:Einleitung}

\subsection{Vorstellung der eigenen Person} 
\label{sec:eigene Person}
Mein Name ist Lukas Klettke. Ich bin am 14.01.2001 in Lübeck geboren und in Bad Schwartau aufgewachsen. Dort habe ich die Grundschule und das Leibniz Gymnasium besucht. Nach zwölf Jahren Schulzeit habe ich meinen Schulweg im Jahre 2019 mit dem Abitur abgeschlossen.

Direkt nach Abschluss der Schule habe ich im August 2019 eine Ausbildung zum Fachinformatiker für Anwendungsentwicklung begonnen und bin in der Softwareentwicklung für Diagnostikgeräte tätig.

In meiner Freizeit bin ich als ehrenamtlicher Schwimmtrainer tätig, segle und fahre Rennrad.

\subsection{Vorstellung des Ausbildungsbetriebs} 
\label{sec:Ausbildungsbetrieb}
Mein Ausbildungsbetrieb ist die {\betriebName} mit Sitz in {\betriebOrt} und Zweigstellen in Groß Grönau, Selmsdorf und Dassow im Norden und Rennersdorf, Pegnitz und Bernstadt im Süden Deutschlands. Durch den Verkauf der Firma im Dezember 2017 befindet sich {\betriebNameKzf} in Besitz von {\mutterBetriebName}, einem US-amerikanischen Technologieunternehmen im Bereich der Chemie- und Medizintechnik.

{\betriebNameKzf} ist ein Hersteller für diverse medizinische Diagnostika von Autoimmun-, Infektionskrankheiten und Allergien, aber auch im Bereich der Automatisierung. Die Ausbildung findet in Dassow in der Forschung und Entwicklung von Software zur Steuerung von Diagnostikautomaten statt.

Insgesamt hat {\betriebNameKzf} mehr als 3.200 Mitarbeiter in 17 Ländern.

\subsection{Projektauslöser} 
\label{sec:Projektauslöser}
Neben der Herstellung von medizinischen Diagnostika zur manuellen Durchführung, werden Geräte zur automatisierten Durchführung dessen hergestellt und vertrieben. Diese Diagnostikautomaten arbeiten mit diversen unterschiedlichen Betriebsmitteln (z.B. Reinigungsflüssigkeit zur Reinigung der Schläuche, Probenträger, etc.), welche ebenfalls von {\betriebNameKzf} an die Kunden verkauft werden.

Anhand der verbrauchten Betriebsmittel der Geräte, wird die Menge, der in Zukunft benötigten, berechnet und die Preise dementsprechend auf den Kunden angepasst. Außerdem wird je nach Verbrauch der Labore die Produktions- und Lagermenge optimiert.

Derzeit werden jedoch keine Daten der Verbräuche von den Gerätesoftwares erhoben, was eine manuelle Berechnung derer zur Folge hat. Diese Berechnung wird durch Außendienstmitarbeiter durchgeführt, welche die Labore besuchen und die durchgeführten Testmengen als Maßstab nutzen. Je nach Art des Gerätebetriebs ist der Verbrauch jedoch unterschiedlich: Werden z.B. 500 Tests am Stück durchgeführt, ist das Verhalten des Geräts ein Anderes, als wenn über eine Zeit von zwei Wochen 500 Tests durchgeführt werden. Somit ist bei der Berechnung eine gewisse Ungenauigkeit vorhanden, die in Zusammenhang mit einer großen Anzahl an Kunden, Differenzen zwischen berechneten und reellen Verbräuchen verursacht. 

Durch eine automatisierte und genauere Berechnung mithilfe von protokollierten Ressourcenverbräuchen könnten Punkte wie Preisgestaltung, Produktionsmenge oder Lagerhaltung weiter optimiert werden und somit Geld einsparen bzw. Gewinn maximieren.

\subsection{Projektumfeld}
\label{sec:Projektumfeld}
Das Projektumfeld ist der {\betriebNameKzf} Standort in Dassow. Dort befindet sich ein Teil der Entwicklung der Diagnostikgeräte und der zugehörigen Software. 

Bei diesem Projekt handelt es sich um eine Software, die ausschließlich intern eingesetzt werden soll.

\subsection{Projektziel}
\label{sec:Projektziel}
Ziel des Projekts ist es eine Schnittstelle zu definieren, die unabhängig von Diagnostikgerät und der entsprechenden Software implementiert werden kann. Durch diese Schnittstelle wird definiert, in welcher Form die Verbrauchsdaten abgefragt und verarbeitet werden.

Anhand dessen wird eine Software geschrieben, die die Ressourcenverbräuche verarbeitet, eine Gesamtberechnung durchführt und den Export einer \glqq .xlsx\grqq \xspace ermöglicht, um die nachstehende Kalkulation mittels Microsoft Excel zu gewährleisten.

Die Planung eines solchen Projekts existiert bereits mehrere Jahre und wurde von der Geschäftsführung in Auftrag gegeben. Das Ziel dessen ist es Daten über die Nutzung der {\betriebNameKzf} Diagnostikgeräte zur erheben, welche zur Analyse von weiteren Optimierungmöglichkeiten dienen.

\subsection{Projektschnittstellen}
\label{sec:Projektschnittstellen}
Das Projekt stellt eine {\acs{gRPC}} Schnittstelle bereit. Diese wird mithilfe einer \glqq .proto\grqq \xspace Datei definiert. Die Schnittstelle wird auf Seite des Clients implementiert und auf Seite des Servers offen gelassen, sodass die unterschiedlichen Softwares der Diagnostikgeräte diese implementieren können und die Freiheit haben, je nach Architektur und Speicherung, die Daten bereitzustellen.

Des Weiteren wird der Export einer \glqq .xlsx\grqq \xspace Datei angeboten.