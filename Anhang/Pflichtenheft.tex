\subsection{Pflichtenheft (Auszug)}
\label{app:Pflichtenheft}

\subsubsection*{Zielbestimmung}
\begin{enumerate}[itemsep=0em,partopsep=0em,parsep=0em,topsep=0em]
	\item Musskriterien % Wikipedia: für das Produkt unabdingbare Leistungen, die in jedem Fall erfüllt werden müssen
	\begin{enumerate}
		\item Modul-Liste: Zeigt eine sortierbare Liste der Betriebsmittelverbräuche mit den dazugehörigen Kerninformationen
		\begin{itemize}
			\item In der Liste wird der Name des Verbrauches, das entsprechende Gerät und die Zeit angezeigt.
			\item Die einzelnen Listeneinträge können zur weiteren Berechnung ausgewählt werden, was mittels einer CheckBox angezeigt wird.
			\item Die Liste ist nach der Zeit des Betriebsmittelverbrauchs filterbar.
		\end{itemize}
		\item Liste der berechneten Betriebsmittelverbräuche: Zeigt eine sortierbare Liste der zusammengefassten Verbräuche
		\begin{itemize}
			\item Die Verbräuche werden nach Namen gegliedert.
			\item Innerhalb des Verbrauchs werden gleichnamige Betriebsmittel mit gleichen Einheiten zusammengerechnet.
		\end{itemize}
		\item Export der berechneten Betriebsmittelverbräuche
		\begin{itemize}
			\item Zur weiteren Kalkulation mit Excel wird der Export einer \glqq .xlsx\grqq \space Datei bereit gestellt.
			\item Die Auflistung in der Datei wird wie oben genannt statt finden.
		\end{itemize}
		\item Sonstiges
		\begin{itemize}
			\item Es wird eine Auflistung aller verbundenen Diagnostikgeräte geben.
			\item Die Anwendung soll möglichst leicht erweiterbar sein und auch von anderen Entwicklungsprozessen ausgehen können.
		\end{itemize}
	\end{enumerate}
\end{enumerate}
\subsubsection*{Produkteinsatz}
\begin{enumerate}[itemsep=0em,partopsep=0em,parsep=0em,topsep=0em]
	\item Anwendungsbereiche

	Die Client-Anwendung wird von Außendienstmitarbeitern genutzt, um in Laboren mit {\betriebNameKzf} Diagnostikgeräten gesammelt die Betriebsmittelverbräuche abzufragen.
\end{enumerate}