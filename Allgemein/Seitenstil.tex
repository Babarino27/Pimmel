% !TEX root = ../Projektdokumentation.tex

% Seitenränder -----------------------------------------------------------------
\setlength{\topskip}{\ht\strutbox} % behebt Warnung von geometry
\geometry{a4paper,left=20mm,right=20mm,top=25mm,bottom=35mm}

\usepackage[
	automark, % Kapitelangaben in Kopfzeile automatisch erstellen
	headsepline,
	ilines
]{scrlayer-scrpage}

% Inhaltsverzeichnis
\DeclareTOCStyleEntries[
dynindent
]{tocline}{section,subsection,subsubsection,paragraph,subparagraph, figure, table}

% Tabellenverzeichnis
\usepackage{tocloft}
\renewcommand{\cfttabpresnum}{Tabelle~}
\newlength{\tabnumw}\settowidth{\tabnumw}{\bfseries\sffamily Tabelle~99}

% Kopf- und Fußzeilen ----------------------------------------------------------
\pagestyle{scrheadings}

% Kopfzeile
\renewcommand{\headfont}{\normalfont} % Schriftform der Kopfzeile
\ihead{\large{\textsc{\titel}}\\ \small{\untertitel} \\[2ex] \textit{\headmark}}
\chead{}
\ohead{\includegraphics[scale=1.05]{\betriebLogo}}
\setlength{\headheight}{10mm} % Höhe der Kopfzeile
%\setheadwidth[0pt]{textwithmarginpar} % Kopfzeile über den Text hinaus verbreitern (falls Logo den Text überdeckt)

% Fußzeile
\ifoot{\autorName}
\cfoot{}
\ofoot{\pagemark}

\newcommand{\headingSpace}{1.5cm}

\renewcommand*{\othersectionlevelsformat}[3]{
  \makebox[\headingSpace][l]{#3\autodot}
}

\titleformat{\section}
{\normalfont\sffamily\Large\bfseries}
{\thesection}{1em}{}

\cftsetindents{section}{0.0cm}{\headingSpace}
\cftsetindents{subsection}{0.0cm}{\headingSpace}
\cftsetindents{subsubsection}{0.0cm}{\headingSpace}
\cftsetindents{figure}{0.0cm}{\headingSpace}
\cftsetindents{table}{0.0cm}{\headingSpace}

\onehalfspacing % Zeilenabstand 1,5 Zeilen
\frenchspacing % erzeugt ein wenig mehr Platz hinter einem Punkt

\clubpenalty = 10000
\widowpenalty = 10000
\displaywidowpenalty = 10000

\counterwithout{footnote}{section} % Fußnoten fortlaufend durchnummerieren
\setcounter{tocdepth}{3} % im Inhaltsverzeichnis werden die Kapitel bis zum Level der subsubsection übernommen
\setcounter{secnumdepth}{3} % Kapitel bis zum Level der subsubsection werden nummeriert

% Aufzählungen anpassen
\renewcommand{\labelenumi}{\arabic{enumi}.}
\renewcommand{\labelenumii}{\arabic{enumi}.\arabic{enumii}.}
\renewcommand{\labelenumiii}{\arabic{enumi}.\arabic{enumii}.\arabic{enumiii}}

% Tabellenfärbung:
\definecolor{heading}{rgb}{0.64,0.78,0.86}
\definecolor{odd}{rgb}{0.9,0.9,0.9}